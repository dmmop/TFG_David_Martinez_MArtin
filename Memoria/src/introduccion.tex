\chapter{Introducción \label{sec:intro}}

A consecuencia del proceso de transformación digital que lleva produciéndose desde hace unos años de forma general en la sociedad y, especialmente, en el mundo empresarial, el tráfico y la generación de datos ha aumentado de forma muy abrupta, generándose millones de terabytes de nuevos datos al día \cite{BDStats}. 


La procedencia de estos datos es muy variada, sin embargo, la irrupción en la vida cotidiana de los dispositivos conectados a la red ha sido la principal razón del crecimiento exponencial que se ha producido en estos últimos años. Concretamente, son el crecimiento en el número de smartphones \cite{phoneGrowth} y, más recientemente, la llegada del Internet de las cosas (IOT) y los wearable los principales responsables de este crecimiento en la cantidad de datos generados. 

Este gran volumen de datos hace que la tarea del procesado de los mismos haya pasado a ser mucho más costosa. Además, el uso de diferentes fuentes de información hace que su estructura no sea homogénea, de forma que aumenta la complejidad del procesado. Esto es debido a la necesidad de su reestructuración para poder ser utilizados.

Los cambios en la manera de obtener los datos han conllevado a que nuevas herramientas hayan sido necesarias para suplir las carencias e incapacidades que las tradicionales tenían para procesar los datos. El desarrollo de estas nuevas tecnologías ha supuesto un desafío para las empresas y organizaciones, que buscan obtener ventajas competitivas mediante el estudio de toda la información.

Es decir, las organizaciones y empresas buscan, con el desarrollo de estos sistemas, obtener de los datos la información más valiosa posible para sus negocios y, así, lograr ventaja con respecto a sus competidores y poder trazar estrategias que aumenten los beneficios.

Por otro lado, hay organizaciones y entes públicos que están utilizando estas herramientas de procesamiento y análisis para mejorar el funcionamiento de diversos sistemas. Especial importancia está tomando procesado en tiempo real que permiten mostrar la información de forma instantánea y predecir tendencias en cortos espacios de tiempo, por ejemplo, en los sistemas de transporte, permitiendo la toma de medidas para optimizar su funcionamiento. 

Este proyecto de fin de grado analizará diferentes fuentes de información de uso común para programadores de forma profesional y personal con el fin de obtener una conclusión que permita una toma de decisiones basada en los datos obtenidos, permitiendo una mejora de las decisiones así como una reducción en los costes y una mayor captación de clientes potenciales como resultado directo este proyecto.

En este documento se expondrán el diseño implementado para el sistema \textit{big data}, que será desarrollado con las tecnologías \textit{Apache Spark} \cite{spark}, presente en todas las implementaciones realizadas, y con \textit{Apache Hadoop} \cite{hadoop}, cuyo sistema distribuido de ficheros será utilizado en conjunción con el primero en algún diseño.




\section{Objetivos}

El principal objetivo es realizar un informe donde se indiquen los lenguajes de programación más populares, el idioma vehicular de los programadores y los paises donde estos tienen mayor impacto. Para alcanzar el objetivo, el proyecto contará con las siguientes fases de trabajo:

\begin{itemize}
	\item Análisis y estudio del panorama y de las herramientas \textit{big data} disponibles en la actualidad.
	
	\item Estudio de los insights a resolver y las fuentes de datos.
	
	\item Diseño del sistema \textit{big data}.
	
	\item Implementación del sistema \textit{big data} diseñado.
	
	\item Aprovisionamiento, analisis y procesado de datos.
	
	\item Obtención de las conclusiones. 
\end{itemize}

\clearpage
\section{Estructura del documento}

En este apartado se presentará la estructura que seguirá el documento con el fin de ofrecer una visión general del mismo y facilitar su lectura y seguimiento.

\begin{enumerate}
	\item \textbf{Introducción:} Donde se presenta el contexto en el que se realiza el proyecto realizado, así como la motivación del mismo, los objetivos y la estructura del documento. Apartado \ref{sec:intro}.
	
	\item \textbf{Estado del arte:} Donde se analiza el contexto en el que se realiza el proyecto respecto a las tecnologías y herramientas utilizadas, en este caso el \textit{big data} y el open data. Apartado \ref{sec:estado_del_arte}.
	
	\item \textbf{Marco regulador:} Detalla las leyes y estándares que se aplican en los proyectos \textit{big data} en la realidad. Apartado \ref{sec:MarcoRegulador}.

	\item \textbf{Diseño:} En este apartado se describe el proceso de diseño del sistema y los objetivos a cumplir. Apartado \ref{sec:disenho}.

	\item \textbf{Instalación:} Se describe el proceso de implementación de los sistemas diseñados, describiendo los pasos necesarios y realizados para hacer funcionar al sistema. Apartado \ref{sec:implementacion}.

	\item \textbf{Stack Overflow:} Detalla los pasos realizados para el aprovisionamiento, análisis y procesado de datos de esta plataforma. Apartado \ref{sec:stackoverflow}.
	
	\item \textbf{Twitter:} Detalla los pasos realizados para el aprovisionamiento, análisis y procesado de datos de esta plataforma. Apartado \ref{sec:twitter}.
	
	\item \textbf{GitHub:} Detalla los pasos realizados para el aprovisionamiento, análisis y procesado de datos de esta plataforma. Apartado \ref{sec:github}.
	
	\item \textbf{Conclusiones:} Se realiza una retrospectiva sobre el trabajo realizado, así como un informe final donde se da respuesta a los objetivos planteados. Apartado \ref{sec:conclusiones}.
	
	\item \textbf{Apéndices:} Información extra sobre el proyecto.

\end{enumerate}